% are there any problems with removing the time-steps from iterators??
\documentclass{beamer}

\usefonttheme{serif}
\usepackage{graphicx} % Allows including images
\usepackage{listings}
\usepackage{booktabs} % Allows the use of \toprule, \midrule and \bottomrule in tables

\usepackage{enumitem}
\usepackage{bussproofs}
\usepackage{amsmath}
\usepackage{amsthm}
\usepackage{amsfonts}
\usepackage{latexsym}
\setbeamertemplate{footline}[frame number]

\title[WebAssembly Primer]{A WebAssembly Primer}
%% \title{Indexed Reactive Programming}

\author{
Alexandre St-Louis Fortier \\
Stefan Knudsen \\
Cheuk Chuen Siow }
\institute[McGill] {COMP 520 - Group 14}
\date{\today}

\begin{document}
\lstset{language=ML}

\begin{frame}
\titlepage % Print the title page as the first slide
\end{frame}

%% \begin{frame}
%% \frametitle{Overview} % Table of contents slide, comment this block out to remove it
%% \tableofcontents % Throughout your presentation, if you choose to use \section{} and \subsection{} commands, these will automatically be printed on this slide as an overview of your presentation
%% \end{frame}

%% \section{Basic Types}

\begin{frame}
\frametitle{}
\begin{block}{}
  WebAssembly is a low-level programming language designed to be efficient and faster than JavaScript.
\end{block}

\end{frame}

%------------------------------------------------

\begin{frame}
\frametitle{}
\begin{itemize}
\item WebAssembly has a binary format, \texttt{.wasm}, and an abstract tree format, \texttt{.wast}
\item We used the abstract tree format, which has a syntax based on S-expressions
\end{itemize}
\end{frame}

%------------------------------------------------

\begin{frame}
\frametitle{}

\begin{itemize}
  \item Within a function, all local variables must be declared at the beginning
  \item The following is the syntax that we used
\end{itemize}
\end{frame}


%------------------------------------------------

\begin{frame}[fragile]
\frametitle{Syntax}
\begin{verbatim}
var: <int> | $<name>
name: (<letter> | <digit> | _ | . | + | - | *
| / | \ | ^ | ~ | = | < | > | ! | ? | @ | #
| $ | % | & | | | : | ' | `)+
string: ``(<char> | \n | \t | \\
| \' | \" | \<hex><hex>)*''
\end{verbatim}
\end{frame}

%------------------------------------------------

\begin{frame}[fragile]
\frametitle{Types}

\begin{verbatim}
 value: <int> | <float>
 type: i32 | i64 | f32 | f64
\end{verbatim}

%% \begin{itemize}
%%   \item We used i32 and f64
%%   %% \item Since webassembly is an assembly language, we aren't given strings
%% \end{itemize}
\end{frame}

%------------------------------------------------

\begin{frame}[fragile]
\frametitle{Operators}
\begin{verbatim}
binop: eq | ne | lt_s | le_s | gt_s | ge_s | add
| sub | or | and | xor | mul | div_s | rem_s | shl_s
| shr_s | and | or | shr_s
unop:
\end{verbatim}
We ended up defining our GoLite unary primitives in terms of the binary operations
\end{frame}

%------------------------------------------------

\begin{frame}[fragile]
\frametitle{Expressions}

\begin{verbatim}
expr: ( block <name>? <expr>* )
| ( loop <name1>? <name2>? <expr>* )         
;; = (block <name1>? (loop <name2>? (block <expr>*)))
| ( if <expr> ( then <name>? <expr>* )
	( else <name>? <expr>* )? )
| ( if <expr1> <expr2> <expr3>? )            
;; = (if <expr1> (then <expr2>) (else <expr3>?))
| ( br <var> <expr>? )
| ( return <expr>? )                         
;; = (br <current_depth> <expr>?)

\end{verbatim}

\end{frame}

%------------------------------------------------

\begin{frame}[fragile]
\frametitle{More expressions}

\begin{verbatim}
expr: ( call <var> <expr>* )
| ( get_local <var> )
| ( set_local <var> <expr> )
| ( <type>.load((8|16|32)_<sign>)?
	<offset>? <align>? <expr> )
| ( <type>.store(8|16|32)? <offset>?
	<align>? <expr> <expr> )
| ( <type>.const <value> )
| ( <type>.<binop> <expr> <expr> )

\end{verbatim}

\end{frame}

%------------------------------------------------


\begin{frame}[fragile]
\frametitle{}

\begin{verbatim}
func:   ( func <name>? <type>? <param>*
	<result>? <local>* <expr>* )
param:  ( param <type>* ) | ( param <name> <type> )
result: ( result <type> )
local:  ( local <type>* ) | ( local <name> <type> )

\end{verbatim}

\end{frame}

%------------------------------------------------


\begin{frame}[fragile]
\frametitle{Header}

We used the following functions for the \texttt{gen.ml} header:
\begin{verbatim}
expr:
( br_if <var> <expr>? <expr> )
( call_import <var> <expr>* )
module:  ( module <type>* <func>* <import>* <export>*
	<table>* <memory>? <start>? )
import:  ( import <name>? <string> <string>
	(param <type>* ) (result <type>)* )
start:   ( start <var> )
memory:  ( memory <int> <int>? <segment>* )

\end{verbatim}

\end{frame}

\begin{frame}
\frametitle{References}
\url{https://github.com/WebAssembly/spec/blob/master/ml-proto/README.md}
\end{frame}

\end{document} 


SYNTAX




